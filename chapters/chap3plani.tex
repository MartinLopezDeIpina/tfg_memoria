En este capítulo se presenta la planificación del proyecto, abordando el alcance definido, los periodos de realización de tareas y los diversos ámbitos de gestión: temporal, de riesgos, de comunicaciones e información, y de partes interesadas. El objetivo de esta planificación es establecer una hoja de ruta estructurada que permita el cumplimiento de todos los objetivos del proyecto dentro de los plazos establecidos.

\section{Alcance}

Este proyecto aborda el desarrollo de un sistema basado en agentes LLM implementado sobre una solución software propia de LKS NEXT, con el propósito de investigar el comportamiento de diversas arquitecturas de agentes definidas en el capítulo \ref{ch:chap2} y evaluar su viabilidad para su futura incorporación en los procesos productivos de la empresa.

En dicho sistema, se implementarán diversas modalidades de interacción entre agentes especializados en fuentes de datos específicas, evaluando la eficacia de los distintos patrones de comunicación entre dichos componentes.

Aun así, el alcance del proyecto no está definido en su totalidad, dada la complejidad de estimación inherente a su naturaleza exploratoria y al uso de tecnologías emergentes. Para mitigar riesgos en el desarrollo, se ha establecido un protocolo preventivo que contempla reuniones quincenales de seguimiento y control.

\subsection{Objetivos concretos del proyecto}

Para facilitar el logro del objetivo principal, se han identificado y definido los siguientes objetivos específicos que estructuran el avance progresivo del proyecto:

\begin{itemize}
  \item\textbf{Estudio de arquitecturas agénticas: }Realizar un análisis de las diversas arquitecturas de agentes, considerando distintas estrategias de interacción y mecanismos de acceso a fuentes de información.
  \item\textbf{Estudio del Model Context Protocol: }Investigar las características y beneficios que aporta la implementación del protocolo MCP, con el objetivo de realizar una valoración objetiva para su posible integración en el entorno profesional de la empresa.
  \item\textbf{Evaluación de agentes: }Desarrollar un sistema de evaluación para proporcionar métricas comparativas cuantificables sobre el rendimiento de los diferentes enfoques de agentes. 
  \item\textbf{Desarrollo de sistema de Onboarding: }Implementar las arquitecturas propuestas en un proyecto software corporativo, con el propósito de analizar su eficacia en la asistencia a nuevos integrantes durante su proceso de incorporación a la empresa.
  \item\textbf{Valoración de ajuste de agentes: }Analizar la relación coste-beneficio asociada al proceso de ajuste fino de modelos LLM para su aplicación en agentes concretos.
\end{itemize}

Adicionalmente, el objetivo es desarrollar el sistema de onboarding incorporando en la medida de los posible en su base de conocimiento la metodología de trabajo implementada en la empresa. Mediante la adhesión a los estándares definidos en un entorno profesional real, se pretende garantizar que los resultados obtenidos constituyan un reflejo de la viabilidad de implementación y eficacia de proyectos similares en un sistema de explotación. 

\subsection{Requisitos}
Los requisitos del proyecto se detallan en profundidad en la sección \ref{}

\subsection{Fases del proyecto}
Tal y como se ha mencionado anteriormente, el alcance del proyecto no está completamente definido debido a su naturaleza exploratoria. Consecuentemente, se propone un ciclo de vida iterativo-incremental con iteraciones de aproximadamente dos semanas de duración, permitiendo una adaptación progresiva a los requisitos emergentes.

La primera iteración se centra en la captura de requisitos del proyecto, donde se explorará y definirá el alcance del sistema de agentes a desarrollar. Tras establecer estas bases, la segunda iteración abordará la implementación de un sistema de agentes mínimo que contenga la estructura general del sistema, proporcionando un marco operativo inicial.

Con este sistema mínimo implementado, la tercera iteración corresponderá al desarrollo de un mecanismo de evaluación, con el objetivo de establecer métricas que permitan mejorar el sistema en la cuarta iteración, donde se aplicarán las optimizaciones identificadas. La quinta iteración se dedicará a la implementación de arquitecturas de agentes exploratorias, evaluando su rendimiento mediante las métricas previamente establecidas.

Finalmente, la sexta iteración contemplará el ajuste fino de un modelo LLM para su integración en un agente específico del sistema, así como su evaluación contextualizada dentro del marco operativo general.

Gracias a este tipo de ciclo de vida, se alcanzan los objetivos del proyecto de manera progresiva, obteniendo retroalimentación directa por parte de los directores del proyecto en cada iteración, lo que permite redirigir la dirección del trabajo si fuera necesario ante posibles contratiempos.

\subsection{Descomposición de tareas}
La Estructura de Descomposición de Trabajo (EDT) del proyecto se ha creado considerando el ciclo de vida iterativo-incremental del proyecto. La figura \ref{} ilustra un diagrama de esta. 

Se divide en los siguientes paquetes:

\begin{itemize}
  \item\textbf{1ª iteración:}
    \begin{itemize}
      \item\textbf{Captura de requisitos (CR): }Conjunto de paquetes de trabajo que capturan o generan los recursos necesarios para el desarrollo del proyecto. 
            \begin{itemize}
          \item\textbf{Definición de requisitos principales: } Captura de requisitos respecto al alcance general del proyecto, validados con los directores del proyecto al comienzo de este.
          \item\textbf{Elicitación de requisitos: }Recopilación de preguntas potenciales para el sistema mediante un cuestionario electrónico y su posterior procesamiento.
        \end{itemize}
        \begin{itemize}
          \item\textbf{Selección y generación de recursos: }Selección del proyecto software y documentación sobre la que desarrollar el sistema, así como la generación de documentación extra.
        \end{itemize}
    \end{itemize}
    \begin{itemize}
      \item\textbf{Definición del alcance y requisitos del sistema de agentes (DA): }Conjunto de paquetes de trabajo que acotan el alcance del sistema implementado considerando trabajo externo previamente realizado.
        \begin{itemize}
          \item\textbf{Investigación de arquitecturas del estado del arte: }Exploración de arquitecturas investigadas por la comunidad académica.
        \end{itemize}
        \begin{itemize}
          \item\textbf{Búsqueda de proyectos parecidos: }Exploración de implementaciones de sistemas parecidos en proyectos software y Onboarding.
        \end{itemize}
    \end{itemize}
  \item\textbf{2ª iteración:}
    \begin{itemize}
      \item\textbf{Sistema de agentes (SA): }Conjunto de paquetes de trabajo que se centran en el diseño, implementación y evaluación de un sistema de agentes LLM para la asistencia en un proyecto software.
        \begin{itemize}
          \item\textbf{Diseño del sistema: }Diseño e implementación mínima de los diferentes módulos del sistema. 
        \end{itemize}
        \begin{itemize}
          \item\textbf{Implementación de agentes especializados: }Desarrollo de varios agentes especializados en las fuentes de información disponibles. 
        \end{itemize}
        \begin{itemize}
          \item\textbf{Implementación de sistema de comunicación mínima: }Crear un sistema de orquestación básico para los agentes implementados.         
        \end{itemize}
    \end{itemize}
  \item\textbf{3ª iteración:}
    \begin{itemize}
      \item\textbf{Sistema de agentes (SA): }Conjunto de paquetes de trabajo que se centran en el diseño, implementación y evaluación de un sistema de agentes LLM para la asistencia en un proyecto software.
        \begin{itemize}
          \item\textbf{Desarrollo del sistema de evaluación}: Implementación de un mecanismo de evaluación automático sobre el sistema mínimo.
          \item\textbf{Captura de datos de evaluación}: Anotación manual de ejemplos para la evaluación del rendimiento del sistema impelementado. 
        \end{itemize}
    \end{itemize}
  \item\textbf{4ª iteración:}
    \begin{itemize}
      \item\textbf{Sistema de agentes (SA): }Conjunto de paquetes de trabajo que se centran en el diseño, implementación y evaluación de un sistema de agentes LLM para la asistencia en un proyecto software.
        \begin{itemize}
          \item\textbf{Mejora de agentes implementados: }Modificación de los agentes mínimos implementados para su mejora en las métricas definidas.
          \item\textbf{Variaciones de mecanismos de orquestación: }Exploración de estrategias de orquestación alternativas al sistema mínimo implementado. 
        \end{itemize}
    \end{itemize}
  \item\textbf{5ª iteración:}
    \begin{itemize}
      \item\textbf{Sistema de agentes (SA): }Conjunto de paquetes de trabajo que se centran en el diseño, implementación y evaluación de un sistema de agentes LLM para la asistencia en un proyecto software.
        \begin{itemize}
          \item\textbf{Exploración de arquitecturas de interacción alternativas: }Implementación y evaluación de mecanismos adicionales de interacción y orquestación agénticas.
          \item\textbf{Integración de módulos de memoria: }Adición de un mecanismo de memoria y la evaluación de su eficacia en el sistema. 
          \item\textbf{Implementación de agentes avanzados: }Desarrollo e integración de agentes con un proceso de ejecución extensa para su evaluación de costo-beneficio. 
        \end{itemize}
    \end{itemize}
  \item\textbf{6ª iteración:}
    \begin{itemize}
      \item\textbf{Ajuste de modelo (AM): }Conjunto de paquetes de trabajo centrados en el entrenamiento de un modelo para un agente específico del sistema.
      \begin{itemize}
        \item\textbf{Selección del agente: }Evaluación justificada del agente específico a ajustar.
        \item\textbf{Extracción de datos: }Registro automático de datos de entrenamiento desde el agente a ajustar, utilizando para ello un modelo de alto rendimiento.
        \item\textbf{Entrenamiento del modelo: }Desarrollo y ejecución del ciclo de entrenamiento del modelo LLM.
      \end{itemize}
      \item\textbf{Sistema de agentes (SA): }Conjunto de paquetes de trabajo que se centran en el diseño, implementación y evaluación de un sistema de agentes LLM para la asistencia en un proyecto software.
        \begin{itemize}
          \item\textbf{Incorporación y evaluación del modelo ajustado: }Desarrollo de adaptadores necesarios para integrar el modelo ajustado al sistema implementado, así como su posterior evaluación.
        \end{itemize}
    \end{itemize}
  \item\textbf{Gestión (G): }
    \begin{itemize}
      \item\textbf{Planificación (P): }Proceso de establecimiento de directrices, objetivos y actividades para el desarrollo exitoso del proyecto.
      \item\textbf{Seguimiento y control (SC): }Supervisión de avances con reuniones bisemanales con los directores del proyecto.
    \end{itemize}
  \item\textbf{Trabajo académico (TA): }
    \begin{itemize}
      \item\textbf{Memoria (M): }Redacción de la memoria del proyecto. 
      \item\textbf{Defensa (D): }Elaboración y preparación de la defensa del proyecto.
    \end{itemize}
\end{itemize}






