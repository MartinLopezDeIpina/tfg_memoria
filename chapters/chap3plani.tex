En este capítulo se presenta la planificación del proyecto, abordando el alcance definido, los periodos de realización de tareas y los diversos ámbitos de gestión: temporal, de riesgos, de comunicaciones e información, y de partes interesadas. El objetivo de esta planificación es establecer una hoja de ruta estructurada que permita el cumplimiento de todos los objetivos del proyecto dentro de los plazos establecidos.

\section{Alcance}

Este proyecto aborda el desarrollo de un sistema basado en agentes LLM implementado sobre una solución software propia de LKS NEXT, con el propósito de investigar el comportamiento de diversas arquitecturas de agentes definidas en el capítulo \ref{ch:chap2} y evaluar su viabilidad para su futura incorporación en los procesos productivos de la empresa.

En dicho sistema, se implementarán diversas modalidades de interacción entre agentes especializados en fuentes de datos específicas, evaluando la eficacia de los distintos patrones de comunicación entre dichos componentes.

Aun así, el alcance del proyecto no está definido en su totalidad, dada la complejidad de estimación inherente a su naturaleza exploratoria y al uso de tecnologías emergentes. Para mitigar riesgos en el desarrollo, se ha establecido un protocolo preventivo que contempla reuniones quincenales de seguimiento y control.

\subsection{Objetivos concretos del proyecto}

Para facilitar el logro del objetivo principal, se han identificado y definido los siguientes objetivos específicos que estructuran el avance progresivo del proyecto:

\begin{itemize}
  \item\textbf{Estudio de arquitecturas agénticas: }Realizar un análisis de las diversas arquitecturas de agentes, considerando distintas estrategias de interacción y mecanismos de acceso a fuentes de información.
  \item\textbf{Estudio del Model Context Protocol: }Investigar las características y beneficios que aporta la implementación del protocolo MCP, con el objetivo de realizar una valoración objetiva para su posible integración en el entorno profesional de la empresa.
  \item\textbf{Evaluación de agentes: }Desarrollar un sistema de evaluación para proporcionar métricas comparativas cuantificables sobre el rendimiento de los diferentes enfoques de agentes. 
  \item\textbf{Desarrollo de sistema de Onboarding: }Implementar las arquitecturas propuestas en un proyecto software corporativo, con el propósito de analizar su eficacia en la asistencia a nuevos integrantes durante su proceso de incorporación a la empresa.
  \item\textbf{Valoración de ajuste de agentes: }Analizar la relación coste-beneficio asociada al proceso de ajuste fino de modelos LLM para su aplicación en agentes concretos.
\end{itemize}

Adicionalmente, el objetivo es desarrollar el sistema de onboarding incorporando en la medida de los posible en su base de conocimiento la metodología de trabajo implementada en la empresa. Mediante la adhesión a los estándares definidos en un entorno profesional real, se pretende garantizar que los resultados obtenidos constituyan un reflejo de la viabilidad de implementación y eficacia de proyectos similares en un sistema de explotación. 

\subsection{Requisitos}
Los requisitos del proyecto se detallan en profundidad en la sección \ref{}
