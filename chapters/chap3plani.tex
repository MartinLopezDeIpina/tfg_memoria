En este capítulo se presenta la planificación del proyecto, abordando el alcance definido, los periodos de realización de tareas y los diversos ámbitos de gestión: temporal, de riesgos, de comunicaciones e información, y de partes interesadas. El objetivo de esta planificación es establecer una hoja de ruta estructurada que permita el cumplimiento de todos los objetivos del proyecto dentro de los plazos establecidos.

\section{Alcance}

Este proyecto aborda el desarrollo de un sistema basado en agentes LLM implementado sobre una solución software propia de LKS NEXT, con el propósito de investigar el comportamiento de diversas arquitecturas de agentes definidas en el capítulo \ref{ch:chap2} y evaluar su viabilidad para su futura incorporación en los procesos productivos de la empresa.

En dicho sistema, se implementarán diversas modalidades de interacción entre agentes especializados en fuentes de datos específicas, evaluando la eficacia de los distintos patrones de comunicación entre dichos componentes.

Aun así, el alcance del proyecto no está definido en su totalidad, dada la complejidad de estimación inherente a su naturaleza exploratoria y al uso de tecnologías emergentes. Para mitigar riesgos en el desarrollo, se ha establecido un protocolo preventivo que contempla reuniones quincenales de seguimiento y control.

\subsection{Objetivos concretos del proyecto}

Para facilitar el logro del objetivo principal, se han identificado y definido los siguientes objetivos específicos que estructuran el avance progresivo del proyecto:

\begin{itemize}
  \item\textbf{Estudio de arquitecturas agénticas: }Realizar un análisis de las diversas arquitecturas de agentes, considerando distintas estrategias de interacción y mecanismos de acceso a fuentes de información.
  \item\textbf{Estudio del Model Context Protocol: }Investigar las características y beneficios que aporta la implementación del protocolo MCP, con el objetivo de realizar una valoración objetiva para su posible integración en el entorno profesional de la empresa.
  \item\textbf{Evaluación de agentes: }Desarrollar un sistema de evaluación para proporcionar métricas comparativas cuantificables sobre el rendimiento de los diferentes enfoques de agentes. 
  \item\textbf{Desarrollo de sistema de Onboarding: }Implementar las arquitecturas propuestas en un proyecto software corporativo, con el propósito de analizar su eficacia en la asistencia a nuevos integrantes durante su proceso de incorporación a la empresa.
  \item\textbf{Valoración de ajuste de agentes: }Analizar la relación coste-beneficio asociada al proceso de ajuste fino de modelos LLM para su aplicación en agentes concretos.
\end{itemize}

Adicionalmente, el objetivo es desarrollar el sistema de onboarding incorporando en la medida de los posible en su base de conocimiento la metodología de trabajo implementada en la empresa. Mediante la adhesión a los estándares definidos en un entorno profesional real, se pretende garantizar que los resultados obtenidos constituyan un reflejo de la viabilidad de implementación y eficacia de proyectos similares en un sistema de explotación. 

\subsection{Requisitos}
Los requisitos del proyecto se detallan en profundidad en la sección \ref{}

\subsection{Fases del proyecto}
Tal y como se ha mencionado anteriormente, el alcance del proyecto no está completamente definido debido a su naturaleza exploratoria. Consecuentemente, se propone un ciclo de vida iterativo-incremental con iteraciones de aproximadamente dos semanas de duración, permitiendo una adaptación progresiva a los requisitos emergentes.

La primera iteración se centra en la captura de requisitos del proyecto, donde se explorará y definirá el alcance del sistema de agentes a desarrollar. Tras establecer estas bases, la segunda iteración abordará la implementación de un sistema de agentes mínimo que contenga la estructura general del sistema, proporcionando un marco operativo inicial.

Con este sistema mínimo implementado, la tercera iteración corresponderá al desarrollo de un mecanismo de evaluación, con el objetivo de establecer métricas que permitan mejorar el sistema en la cuarta iteración, donde se aplicarán las optimizaciones identificadas. La quinta iteración se dedicará a la implementación de arquitecturas de agentes exploratorias, evaluando su rendimiento mediante las métricas previamente establecidas.

Finalmente, la sexta iteración contemplará el ajuste fino de un modelo LLM para su integración en un agente específico del sistema, así como su evaluación contextualizada dentro del marco operativo general.

Gracias a este tipo de ciclo de vida, se alcanzan los objetivos del proyecto de manera progresiva, obteniendo retroalimentación directa por parte de los directores del proyecto en cada iteración, lo que permite redirigir la dirección del trabajo si fuera necesario ante posibles contratiempos.

\subsection{Descomposición de tareas}
La Estructura de Descomposición de Trabajo (EDT) del proyecto se ha creado considerando el ciclo de vida iterativo-incremental del proyecto. La figura \ref{} ilustra un diagrama de esta. 

Se divide en los siguientes paquetes:

\begin{itemize}
  \item\textbf{1ª iteración:}
    \begin{itemize}
      \item\textbf{Captura de requisitos (CR): }Conjunto de paquetes de trabajo que capturan o generan los recursos necesarios para el desarrollo del proyecto. 
            \begin{itemize}
          \item\textbf{Definición de requisitos principales: } Captura de requisitos respecto al alcance general del proyecto, validados con los directores del proyecto al comienzo de este.
          \item\textbf{Elicitación de requisitos: }Recopilación de preguntas potenciales para el sistema mediante un cuestionario electrónico y su posterior procesamiento.
        \end{itemize}
      \item\textbf{Recopilación de recursos disponibles (RR): }Conjunto de paquetes de trabajo centrados en recopilar, elegir y generar recursos para el desarrollo del proyecto.
        \begin{itemize}
          \item\textbf{Selección de proyecto software a utilizar: }Selección del proyecto software y documentación sobre la que desarrollar el sistema.
          \item\textbf{Generación de recursos: }Generación de documentación extra para uso del sistema de agentes.
        \end{itemize}
      \item\textbf{Definición del alcance y requisitos del sistema de agentes (DA): }Conjunto de paquetes de trabajo que acotan el alcance del sistema implementado considerando trabajo externo previamente realizado.
    \begin{itemize}
          \item\textbf{Investigación de arquitecturas del estado del arte: }Exploración de arquitecturas investigadas por la comunidad académica.
          \item\textbf{Búsqueda de proyectos parecidos: }Exploración de implementaciones de sistemas parecidos en proyectos software y Onboarding.
    \end{itemize}
      \end{itemize}
  \item\textbf{2ª iteración:}
    \begin{itemize}
      \item\textbf{Sistema de agentes (SA): }Conjunto de paquetes de trabajo que se centran en el diseño, implementación y evaluación de un sistema de agentes LLM para la asistencia en un proyecto software.
        \begin{itemize}
          \item\textbf{Diseño del sistema: }Diseño e implementación mínima de los diferentes módulos del sistema. 
        \end{itemize}
        \begin{itemize}
          \item\textbf{Implementación de agentes especializados: }Desarrollo de varios agentes especializados en las fuentes de información disponibles. 
        \end{itemize}
        \begin{itemize}
          \item\textbf{Implementación de sistema de comunicación mínima: }Crear un sistema de orquestación básico para los agentes implementados.         
        \end{itemize}
    \end{itemize}
  \item\textbf{3ª iteración:}
    \begin{itemize}
      \item\textbf{Sistema de agentes (SA): }Conjunto de paquetes de trabajo que se centran en el diseño, implementación y evaluación de un sistema de agentes LLM para la asistencia en un proyecto software.
        \begin{itemize}
          \item\textbf{Desarrollo del sistema de evaluación}: Implementación de un mecanismo de evaluación automático sobre el sistema mínimo.
          \item\textbf{Captura de datos de evaluación}: Anotación manual de ejemplos para la evaluación del rendimiento del sistema impelementado. 
        \end{itemize}
    \end{itemize}
  \item\textbf{4ª iteración:}
    \begin{itemize}
      \item\textbf{Sistema de agentes (SA): }Conjunto de paquetes de trabajo que se centran en el diseño, implementación y evaluación de un sistema de agentes LLM para la asistencia en un proyecto software.
        \begin{itemize}
          \item\textbf{Mejora de agentes implementados: }Modificación de los agentes mínimos implementados para su mejora en las métricas definidas.
          \item\textbf{Variaciones de mecanismos de orquestación: }Exploración de estrategias de orquestación alternativas al sistema mínimo implementado. 
        \end{itemize}
    \end{itemize}
  \item\textbf{5ª iteración:}
    \begin{itemize}
      \item\textbf{Sistema de agentes (SA): }Conjunto de paquetes de trabajo que se centran en el diseño, implementación y evaluación de un sistema de agentes LLM para la asistencia en un proyecto software.
        \begin{itemize}
          \item\textbf{Exploración de arquitecturas de interacción alternativas: }Implementación y evaluación de mecanismos adicionales de interacción y orquestación agénticas.
          \item\textbf{Integración de módulos de memoria: }Adición de un mecanismo de memoria y la evaluación de su eficacia en el sistema. 
          \item\textbf{Implementación de agentes avanzados: }Desarrollo e integración de un agente dotado de un proceso de ejecución extenso para su evaluación de costo-beneficio. Esto puede involucrar varios saltos de llamadas RAG. 
        \end{itemize}
    \end{itemize}
  \item\textbf{6ª iteración:}
    \begin{itemize}
      \item\textbf{Ajuste de modelo (AM): }Conjunto de paquetes de trabajo centrados en el entrenamiento de un modelo para un agente específico del sistema.
      \begin{itemize}
        \item\textbf{Selección del agente: }Evaluación justificada del agente específico a ajustar.
        \item\textbf{Extracción de datos: }Registro automático de datos de entrenamiento desde el agente a ajustar, utilizando para ello un modelo de alto rendimiento.
        \item\textbf{Entrenamiento del modelo: }Desarrollo y ejecución del ciclo de entrenamiento del modelo LLM.
      \end{itemize}
      \item\textbf{Sistema de agentes (SA): }Conjunto de paquetes de trabajo que se centran en el diseño, implementación y evaluación de un sistema de agentes LLM para la asistencia en un proyecto software.
        \begin{itemize}
          \item\textbf{Incorporación y evaluación del modelo ajustado: }Desarrollo de adaptadores necesarios para integrar el modelo ajustado al sistema implementado, así como su posterior evaluación.
        \end{itemize}
    \end{itemize}
  \item\textbf{Gestión (G): }
    \begin{itemize}
      \item\textbf{Planificación (P): }Proceso de establecimiento de directrices, objetivos y actividades para el desarrollo exitoso del proyecto.
      \item\textbf{Seguimiento y control (SC): }Supervisión de avances con reuniones bisemanales con los directores del proyecto.
    \end{itemize}
  \item\textbf{Trabajo académico (TA): }
    \begin{itemize}
      \item\textbf{Memoria (M): }Redacción de la memoria del proyecto. 
      \item\textbf{Defensa (D): }Elaboración y preparación de la defensa del proyecto.
    \end{itemize}
\end{itemize}

\section{Periodos de realización de tareas e hitos}
En este apartado se detallarán las dependencias entre las diferentes tareas, así como la estimación de duración y fechas de cada una de ellas

\subsection{Dependencias entre tareas}
Las dependencias de los paquetes de trabajo del proyecto requieren una ejecución del trabajo planificada. La figura \ref{} ilustra dichas dependencias.


El proyecto se inicia con una primera iteración centrada en la captura de requisitos. Debido al carácter exploratorio del proyecto, se ha invertido un tiempo significativo en alinear los objetivos con las necesidades empresariales. Esta fase comprende la captura de requisitos (G.CR), la recopilación de recursos disponibles (RR) y la definición del alcance y requisitos del sistema de agentes (G.DA).

La segunda iteración se orienta al desarrollo de un sistema de agentes mínimo viable (G.SA), para construir una base sobre la que ir integrando mejoras en iteraciones posteriores. 

Una vez obtenida una primera versión operativa, la tercera iteración implementa un sistema de evaluación para dicho sistema de agentes permitiendo así valorar tanto el rendimiento global como las mejoras introducidas en iteraciones subsiguientes. 

La cuarta iteración perfecciona el sistema de agentes inicial, con el propósito de optimizar las métricas evaluadas y abordar las deficiencias identificadas.

Posteriormente, la quinta iteración explora arquitecturas de interacción alternativas para el sistema de agentes, incorporando un sistema de memoria.

Para concluir, la sexta y última iteración se dedica al ajuste de un modelo específico para un agente seleccionado del sistema (G.AM).

\subsection{Diagrama de Gantt}

La figura \ref{} ilustra el diagrama de Gantt, donde se puede ver de manera aproximada tanto el desarrollo del proyecto en el tiempo como la dedicación horaria a cada paquete de trabajo.

\subsection{Hitos}

En la tabla \ref{} se detallan los hitos establecidos para el desarrollo del proyecto. La finalización de la fase de implementación ha sido programada para el 31 de mayo, tras lo cual se destinarán dos semanas íntegramente a la elaboración de la memoria hasta el 14 de junio, proporcionando así un margen de 9 días previos a la fecha límite de entrega. Este período de contingencia está planificado para abordar posibles contratiempos que pudieran surgir durante el proceso.

\section{Gestión del tiempo}
Se ha gestionado el tiempo disponible para el proyecto considerando el alcance definido para cumplir todos los objetivos del proyecto. 

\subsection{Estimación de cada tarea}
La estimación específica de cada tarea se puede ver en la tabla \ref{}

\section{Gestión de riesgos}

Debido al enfoque exploratorio del proyecto, la gestión de riesgos cobra especial importancia. Se han identificado los siguientes riesgos con su correspondiente plan de contingencia: 

\begin{itemize}
  \item\textbf{R1- Concurrencia exploratoria: }Dado que el proyecto está enfocado en tecnologías emergentes, existe la posibilidad de que durante el período de desarrollo emerjan iniciativas paralelas que aborden la misma problemática.  

Para abordar este riesgo, se realizará una previa investigación de soluciones existentes antes de implementar cada módulo del sistema, así como un análisis periódico que permita identificar mejoras inspiradas en avances externos.

\item\textbf{R2- Variabilidad del alcance: }Al ser un proyecto exploratorio, con un alcance inicialmente ambiguo, podría sufrir alteraciones no planificadas. Estos contratiempos podrían suponer un sobrecoste horario, lo que obligaría a replanificar el alcance para no sobrepasar ampliamente los recursos disponibles.

Para mitigar este riesgo, se implementará un seguimiento y control mediante reuniones quincenales que permitirán evaluar la correcta evolución de las actividades y, en caso necesario, adoptar medidas correctivas para garantizar la viabilidad del trabajo dentro de los plazos establecidos.


\item\textbf{R3- Dependencia de sistemas externos: }La implementación del proyecto depende significativamente de sistemas externos, tanto por los modelos de lenguaje accedidos a través de APIs como por los servicios proporcionados por los servidores MCP. Cualquier alteración o interrupción en estos servicios podría comprometer la funcionalidad del sistema implementado.

Para prevenir este riesgo, se diseñará un sistema con un manejo de excepciones robusto que garantice la continuidad operativa incluso cuando alguno de los módulos experimente fallos. Complementariamente, se adoptará una arquitectura de desarrollo modular que facilite la integración o desacoplamiento de componentes según las necesidades evolutivas del proyecto.

\item\textbf{R4- Filtrado de credenciales: }El acceso a recursos externos sobre el que se desarrolla el proyecto requiere de claves secretas. Es fundamental considerar que existen algoritmos de rastreo automatizados que analizan plataformas públicas en busca de credenciales expuestas para su uso fraudulento. La publicación inadvertida de estas claves en entornos públicos podría ocasionar pérdidas económicas considerables o comprometer la seguridad del sistema corporativo.

Para contrarrestar este riesgo, se implementará una política de gestión de credenciales mediante variables de entorno, evitando su inclusión directa en el código fuente o su transferencia a plataformas en la nube. Adicionalmente, se mantendrán todos los repositorios y sistemas posibles en modo de visibilidad privada.

\item\textbf{R5- Pérdida de recursos: }El desarrollo del proyecto se fundamenta en múltiples recursos esenciales, incluyendo el código fuente del sistema, la documentación de requisitos, los diversos artefactos generados durante el proceso de desarrollo y la propia memoria académica. La pérdida de cualquiera de estos elementos debido a fallos técnicos o incidentes fortuitos podría ocasionar un retraso significativo.

Para mitigar este riesgo, se han implementado los sistemas de información descritos en la sección \ref{sec:sys_info} que garantizan el mantenimiento de copias de seguridad actualizadas diariamente en la nube. Mediante esta estrategia preventiva, cualquier eventualidad que afecte al dispositivo principal de desarrollo supondría, como máximo, la pérdida del trabajo correspondiente a una jornada.
\end{itemize}

\section{Gestión de Comunicaciones e Información}

\subsection{Sistema de información}\label{sec}

La gestión de la información del proyecto se ha estructurado mediante diversos sistemas tecnológicos:
\begin{itemize}
\item \textbf{Repositorio GitHub para código fuente:} El código desarrollado será alojado en un repositorio privado de GitHub.
\item \textbf{Repositorio GitHub para documentación:} La memoria del proyecto, elaborada utilizando LaTeX en el entorno local del alumno, será sincronizada con un repositorio dedicado en GitHub.
\item \textbf{Almacenamiento en Google Drive:} Los diversos recursos, referencias bibliográficas y materiales auxiliares recopilados durante las fases de desarrollo serán almacenados sistemáticamente en esta plataforma.
\end{itemize}
Esta estructura de gestión de la información aporta diversos beneficios al proyecto. Por un lado, facilita la supervisión continua por parte de los directores e implementa un control de versiones organizado que documenta la evolución del trabajo. Por otro lado, garantiza copias de seguridad actualizadas que protegen la integridad de los datos. Adicionalmente, asegura la accesibilidad para todos los interesados.


\subsection{Sistema de comunicación}
La comunicación eficaz entre alumno, director empresarial y directores académicos resulta imprescindible para el correcto seguimiento y control del proyecto. Las herramientas a utilizar son:
\begin{itemize}
\item\textbf{Correo electrónico: }Canal principal para consultas con los directores del proyecto y comunicación formal con otros miembros de la empresa.
\item\textbf{Google Meet: }Plataforma destinada a la realización de reuniones telemáticas entre los participantes.
\item\textbf{Google Chat: }Herramienta complementaria para resolución de consultas rápidas con el director empresarial.
\end{itemize}

\section{Herramientas disponibles}
Para el desarrollo del proyecto se dispone de diversas herramientas que facilitan su implementación y gestión eficiente. A continuación, se detallan las principales:
\begin{itemize}
\item\textbf{IDE PyCharm: }En el marco del paquete educativo proporcionado por GitHub, se tiene acceso a la versión Professional del entorno de desarrollo integrado PyCharm, el cual ofrece un conjunto de funcionalidades y utilidades especializadas para el desarrollo de proyectos en lenguaje Python.
\item\textbf{Asistencia de herramientas de IA: }Se dispone de herramientas de asistencia basadas en inteligencia artificial, concretamente el autocompletador GitHub Copilot y una suscripción al modelo Claude. Esta última contribuye a optimizar tareas relacionadas con la programación, el procesamiento de documentación y la redacción de la presente memoria.
\item\textbf{Claves de acceso a modelos: }La empresa LKS NEXT ha facilitado las credenciales de acceso necesarias para la integración y ejecución de los diferentes modelos LLM utilizados en el proyecto.
\end{itemize}












