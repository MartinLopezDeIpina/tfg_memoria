En este capítulo se presenta la planificación del proyecto, abordando el alcance definido, los periodos de realización de tareas y los diversos ámbitos de gestión: temporal, de riesgos, de comunicaciones e información, y de partes interesadas. El objetivo de esta planificación es establecer una hoja de ruta estructurada que permita el cumplimiento de todos los objetivos del proyecto dentro de los plazos establecidos.

\section{Alcance}

Este proyecto aborda el desarrollo de un sistema basado en agentes LLM implementado sobre una solución software propia de LKS NEXT, con el propósito de investigar el comportamiento de diversas arquitecturas de agentes definidas en el capítulo \ref{ch:chap2} y evaluar su viabilidad para su futura incorporación en los procesos productivos de la empresa.

En dicho sistema, se implementarán diversas modalidades de interacción entre agentes especializados en fuentes de datos específicas, evaluando la eficacia de los distintos patrones de comunicación entre dichos componentes.

Aun así, el alcance del proyecto no está definido en su totalidad, dada la complejidad de estimación inherente a su naturaleza exploratoria y al uso de tecnologías emergentes. Para mitigar riesgos en el desarrollo, se ha establecido un protocolo preventivo que contempla reuniones quincenales de seguimiento y control.

\subsection{Objetivos concretos del proyecto}

Para facilitar el logro del objetivo principal, se han identificado y definido los siguientes objetivos específicos que estructuran el avance progresivo del proyecto:

