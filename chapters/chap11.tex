Completado el ciclo de desarrollo del proyecto, este capítulo ofrece una síntesis reflexiva del trabajo realizado. Se presentan las conclusiones principales, el repaso de los objetivos, y los retos del proyecto. Finalmente, se presentan las líneas futuras relacionadas con el proyecto y el ámbito de investigación.

\section{Conclusiones}
En este apartado se formulan las reflexiones obtenidas respecto al uso de los agentes LLM en el proceso de incorporación a proyectos software. Primero se presenta una valoración global, tras lo que se pasa a concluir sobre el protocolo MCP y los mecanismos de orquestación de agentes. 

\subsection{Líneas generales}
En términos generales, los agentes han demostrado la capacidad de responder con información valiosa a las preguntas formuladas. Constituyen una herramienta valiosa a la hora de buscar y sintetizar sobre una cantidad grande de información, tarea de gran importancia en incorporaciones que requieren de un proceso de aprendizaje. 

Sin embargo, todo poder conlleva una gran responsabilidad. Este tipo de herramientas no están todavía a la altura de considerarse fuentes de información irrefutables. Algunas evaluaciones muestran que en ciertas ocasiones, los agentes no son capaces de encontrar todas las fuentes de información, y qué decir de las citas correctas o alucinaciones. Es por ello que este proyecto propone utilizar dichas herramientas como fuente de información adicionales o herramientas de búsqueda, sin omitir la revisión y reflexión humana.

\subsection{Protocolo MCP}
Este protocolo ha demostrado ser de gran utilidad, incluso 


\subsection{Orquestación de agentes}




\subsection{}

