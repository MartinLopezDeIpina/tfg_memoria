Una vez establecido el objetivo general del proyecto en el capítulo \ref{ch:chap1} y asentados los conceptos con los que se trabajará en el capítulo \ref{ch:chap2}, en el presente capítulo se detallarán los requisitos establecidos para el desarrollo del sistema.

La captura de requisitos se ha estructurado en dos fases: en primera instancia, se acordaron los requisitos principales y el alcance general con los directores del proyecto; posteriormente, se realizó una captura exhaustiva para determinar los dominios de conocimiento que el sistema debe implementar.

\section{Requisitos principales}

El sistema agéntico implementado debe tener las siguientes características para cumplir con las necesidades del director empresarial: 

\subsection{Requisitos funcionales}
\begin{itemize}
\item\textbf{Agentes especializados: }El sistema contemplará al menos 4 agentes cuya labor esté especializada en responder preguntas sobre fuentes de datos específicas.
\item\textbf{Fuentes de información: }Los agentes del sistema contarán con acceso al menos al repositorio de código del proyecto software, a una documentación del proyecto externa y a un sistema de gestión de tareas. Al menos una de dichas fuentes debe ser accedida por un sistema RAG. 
\item\textbf{Orquestación: }Se requiere un agente coordinador cuya función sea decidir a qué agente especializado delegar una acción a realizar.
\item\textbf{Protocolo MCP: }La implementación priorizará el uso del protocolo MCP: Al menos uno de los agentes especializados empleará dicho protocolo para obtener las herramientas a utilizar.
\item\textbf{Evaluación de agentes: }Se implementará un sistema de evaluación objetivo para determinar si la mejora de agentes incrementa la precisión del sistema. Dado un agente y su mejora, será posible determinar cuál de los dos responde mejor a las preguntas realizadas, utilizando al menos una métrica cuantificable.
\item\textbf{Enfoque del sistema: }La finalidad principal del sistema consistirá en responder preguntas sobre un proyecto software, logrando una tasa de precisión mínima del 75\% sobre un conjunto predefinido de al menos 25 preguntas. Esta precisión se medirá utilizando una métrica cuantificable previamente establecida. 
\item\textbf{Librerías utilizadas: }El desarrollo se apoyará en un conjunto de librerías especializadas: LangChain para el control de modelos y prompting (empleándose en al menos el 75\% de las ejecuciones de modelos y prompts), LangGraph para el flujo lógico requiriendo que al menos el 75\% de los agentes del sistema se ejecuten sobre un grafo compilado de dicha librería, y LangSmith como herramienta de monitorización, garantizando que toda llamada a los modelos sea visualizable en dicha plataforma.\end{itemize}
\subsection{Requisitos no funcionales}
\begin{itemize}
\item\textbf{Control de excepciones: }El sistema incorporará un control de excepciones que garantice la ejecución aún si un agente o una herramienta genera una excepción. Se contempla la captura de excepciones de todos los agentes y todas las herramientas utilizadas.
\item\textbf{Control del tiempo: }La ejecución completa del sistema para un caso de uso no excederá los 5 minutos.
\item\textbf{Control del presupuesto: }Se establece un límite máximo de 25 céntimos por ejecución de un caso de uso para el coste de API de modelos.
\item\textbf{Independencia de modelos: }La arquitectura garantizará independencia respecto a un único proveedor de modelos LLM. La sustitución de los modelos utilizados será posible en menos de 5 minutos.
\item\textbf{Modularidad del sistema: }El orquestador funcionará de manera desacoplada respecto a los agentes especializados. La adición o eliminación de un agente especializado requerirá menos de 5 minutos de desarrollo.
\end{itemize}

\section{Dominios de conocimiento}
De acuerdo con los objetivos previamente establecidos en la sección \ref{chap3:objetivos}, es necesario alinear el sistema con la metodología de desarrollo implementada en la organización. Con este propósito, se ha efectuado una recopilación de las posibles interrogantes dirigidas al sistema, procediendo posteriormente a realizar un análisis de la metodología actualmente vigente en el entorno empresarial.

\subsection{Anotación de preguntas}
Para dicho cometido, se ha llevado a cabo una elicitación de requisitos mediante la implementación un cuestionario electrónico dirigido al personal técnico de la sede de Zuatzu en LKS Next. En dicho cuestionario, se solicitó a los profesionales que, acorde a su ámbito de especialización, anotasen las interrogantes que formularían a un sistema de onboarding hipotético.

Se recabaron un total de 63 interrogantes aportadas por 8 profesionales del ámbito del desarrollo software. Posteriormente, dichas cuestiones fueron ampliadas y categorizadas mediante la utilización del modelo Claude, tras lo cual se procedió a una anotación manual para eliminar elementos redundantes. La compilación clasificada final puede consultarse en el listado \ref{listado:preguntas}, donde se han identificado los siguientes dominios de conocimiento:

\begin{itemize}
\item\textbf{Información general: }Objetivo, finalidad y contexto del proyecto.
\item\textbf{Entorno y despliegue: }Entornos de desarrollo y sistemas de despliegue disponibles.
\item\textbf{Gestión del proyecto: }Comunicación y coordinación del equipo, metodología de contribución y gestión de tareas y requisitos.
\item\textbf{Estándares y prácticas: }Estándares de codificación, visuales, de integración y aspectos legales.
\item\textbf{Documentación: }Las fuentes de documentación disponibles para el proyecto y su ubicación.
\item\textbf{Recursos adicionales: }Recursos externos al proyecto como formaciones y documentación de librerías.
\item\textbf{Arquitectura del sistema: }Cuestiones de diseño del código fuente sobre todos los niveles de abstracción del modelo C4\cite{noauthor_c4_nodate}. Este modelo define los niveles de abstracción de un proyecto software en 4 niveles:
\begin{itemize}
\item Actores y sistemas externos con los que interactúa el sistema
\item Contenedores y aplicaciones que componen el sistema
\item Componentes que definen cada contenedor
\item Diagrama de clases e interfaces de cada componente
\end{itemize}
\end{itemize}

\subsection{Metodología empresarial}
La metodología implementada fue consultada directamente con los dearrolladores de la empresa. Al ser LKS Next una consultoría, la metodología implementada depende en gran medida del cliente específico con el que se está trabajando. Esta metodología cumple en su mayoría las siguientes pautas comunes:
\begin{itemize}
  \item\textbf{Gestión del tiempo: }Los integrantes del proyecto utilizan una aplicación propietaria para computar las horas dedicadas al proyecto.
  \item\textbf{Gestión de tareas: }La asignación y monitorización de tareas se realiza mediante aplicaciones especializadas como Jira o Trello.
  \item\textbf{Documentación: }La documentación de los proyectos está dispersa en varias ubicaciones como páginas web dedicadas o dentro del propio repositorio del proyecto.
  \item\textbf{Respositorio del proyecto: }El código fuente del proyecto se gestiona de forma centralizada en un repositorio común, mayoritariamente en repositorios del GitLab propio de LKS Next.
\end{itemize}

Adicionalmente, el flujo de trabajo entre diseñadores y desarrolladores sigue una metodología especializada. En una reunión con un integrante del equipo de diseño \textbf{referencia al acta} se capturó el proceso utilizado:

En primera instancia, se documentan los requisitos visuales a utilizar en un repositorio de Confluence. De esta forma, el equipo de diseño realiza un prototipo visual con herramientas especializadas como Figma. Tras esto, se exportan dichos diseños a maquetas html, los cuales se comparten mediante Google Drive con los desarrolladores del proyecto.

\section{Recursos a utilizar}
Tras definir los requisitos principales y los dominios de conocimeiento necesarios, se ha de definir el contexto específico en el que incorporar el sistema.  Para ello, no es posible utilizar un proyecto de un cliente en producción, dadas las restricciones legales tanto académicas como de inteligencia artificial.

\subsection{Proyecto software}
Se ha seleccionado el proyecto propietario IA-core-tools. Este proyecto constituye herramientas de desarrollo de agentes LLM con acceso a fuentes de información mediante RAG. La aplicación está desarrollada en Flask y proporciona una api para desarrollar agentes de forma interactiva con la interfaz visual, de forma que los desarrolladores de LKS Next la utilicen para crear soluciones para los proyectos en producción.

Se dispone de acceso al repositorio de GitLab del proyecto, con información detallada de todos los usuarios, contribuciones, gestión de incidencias y gestión de tareas.

La metodología de gestión dista ligeramente de los proyectos orientados a clientes. El desarrollador líder es Aritz Galdos (director empresarial del presente proyecto), y realiza reuniones tanto presenciales como virtuales con los contribuyentes del proyecto. Dadas las restricciones legales, no se dispone de acceso a la aplicación de gestión horaria.

\subsection{Recursos generados}

Para la correcta evaluación del sistema, se ha generado documentación lo más fidedigna posible al proyecto. Para ello, se ha utilizado el modelo Claude con toda la información de gestión mencionada como entrada. También se han incluido ficheros clave del repositorio del código fuente. Adicionalmente, se ha refinado manualmente la documentaión generada para eliminar información incorrecta.

En primera instancia, se han generado 3 documentos relacionadas con la sección visual del proyecto: \textit{funcionamiento\_y\_diseño\_interfaz}, \textit{guia\_de\_estilos\_visual} y \textit{limitaciones\_y\_mejoras\_pendientes}. Estos documentos se han añadido a un repositorio de Confluence para simular la documentación mencionada por los diseñadores visuales. 

A continuación, se han generado 13 documentos que simulan la ``documentación oficial'' del proyecto. Estos capturan lo necesario para entender la gestión, diseño y metodología de contribución del proyecto a cualquier nueva incorporación: \textit{arquitectura-software.md}, \textit{despliegue.md}, \textit{equipo-y-comunicacion.md}, \textit{estandares-codigo.md}, \textit{flujos-trabajo.md}, \textit{guia-contribucion.md}, \textit{informacion-cliente.md}, \textit{metodologia.md}, \textit{modelo-negocio.md}, \textit{onboarding.md}, \textit{README.md}, \textit{referencias-tecnicas.md} y \textit{sistema-gestion-tareas.md}.

Finalmente, se ha generado la ``Documentación API'' del proyecto utilizando el agente RepoAgent\cite{luo_repoagent_2024}. Esta documentación contiene la explicación detallada de todas las clases y métodos del proyecto en Python.










