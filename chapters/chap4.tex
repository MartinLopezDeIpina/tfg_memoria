Una vez asentados los conceptos con los que se trabajará en el Capítulo \ref{ch:chap2} y definidos los objetivos del proyecto en el Capítulo \ref{ch:chap3}, en el presente capítulo se detallarán los requisitos establecidos para el desarrollo del sistema.

La captura de requisitos se ha estructurado en tres fases: en primer lugar, la definición de los requisitos principales y el alcance general con los directores del proyecto; en segundo lugar, la identificación de los dominios de conocimiento en los que el sistema debe especializarse; y finalmente, la determinación de los recursos específicos a utilizar para la implementación de la solución.

\section{Requisitos principales}

El sistema agéntico implementado debe tener las siguientes características para cumplir con las necesidades del director en la empresa: 

\subsection{Requisitos funcionales}
\begin{itemize}
\item\textbf{Agentes especializados: }el sistema contemplará un mínimo de 4 agentes cuya labor esté especializada en responder preguntas sobre fuentes de datos específicas.
\item\textbf{Fuentes de información: }los agentes del sistema dispondrán de acceso a un conjunto de recursos que incluirá el repositorio de código del proyecto software, una documentación externa y un sistema de gestión de tareas. Al menos una de estas fuentes deberá ser accedida mediante una implementación RAG.
\item\textbf{Orquestación: }se requiere un agente coordinador cuya función sea decidir a qué agente especializado delegar una acción a realizar.
\item\textbf{Protocolo MCP: }la implementación priorizará el uso del protocolo MCP: al menos uno de los agentes especializados empleará dicho protocolo para obtener las herramientas a utilizar.
\item\textbf{Evaluación de agentes: }se implementará un sistema de evaluación objetivo para determinar si la mejora de agentes incrementa la precisión del sistema. Dado un agente y su mejora, será posible determinar cuál de los dos responde mejor a las preguntas realizadas, utilizando para ello una métrica cuantificable. 
\item\textbf{Enfoque del sistema: }la finalidad principal del sistema consistirá en responder preguntas sobre un proyecto software, logrando una tasa de precisión mínima del 75\% sobre un conjunto predefinido de no menos de 25 preguntas, aprobado por los directores del proyecto. Esta precisión se medirá utilizando una métrica cuantificable previamente establecida.
\item\textbf{Librerías utilizadas: }el desarrollo se sustentará en tres librerías: LangChain para el control de modelos y gestión de prompts (empleada en al menos el 75\% de las ejecuciones), LangGraph para estructurar el flujo lógico (requiriendo que mínimo tres cuartas partes de los agentes operen sobre un grafo compilado de dicha librería), y LangSmith como herramienta de monitorización (garantizando la visualización de todas las llamadas a modelos en la plataforma).
\end{itemize}

Cabe destacar que los requisitos iniciales del director en la empresa contemplaban únicamente una evaluación manual del sistema. La implementación de un sistema de evaluación automatizada, junto con el establecimiento de una precisión mínima cuantificable, se incorporaron posteriormente en la planificación al identificarse como mejoras significativas para el proyecto.
\subsection{Requisitos no funcionales}
\begin{itemize}
\item\textbf{Control de excepciones: }el sistema incorporará un control de excepciones que garantice la ejecución aún si un agente o una herramienta genera una excepción. Se contempla la captura de excepciones de todos los agentes y todas las herramientas utilizadas.
\item\textbf{Control del tiempo: }la ejecución completa del sistema para un caso de uso no excederá los 5 minutos.
\item\textbf{Control del presupuesto: }se establece un límite máximo de 25 céntimos por ejecución de un caso de uso para el coste de API de modelos. El director en la empresa no especificó una cantidad concreta, sino que instó al uso razonable de los recursos; por ello, se ha elegido esta cifra como límite que no debe sobrepasarse.
\item\textbf{Independencia de modelos: }la arquitectura garantizará independencia respecto a un único proveedor de modelos LLM. La sustitución de los modelos utilizados será posible en menos de 5 minutos.
\item\textbf{Modularidad del sistema: }el orquestador funcionará de manera desacoplada respecto a los agentes especializados. La adición o eliminación de un agente especializado requerirá menos de 5 minutos de desarrollo.
\end{itemize}

\section{Dominios de conocimiento}
De acuerdo con los objetivos previamente establecidos en la Sección \ref{chap3:objetivos}, es necesario alinear el sistema con la metodología de desarrollo implementada en la organización. Con este propósito, se ha efectuado una recopilación de las posibles interrogantes que una nueva incorporación plantearía al sistema durante su integración a un proyecto software, procediendo posteriormente a realizar un análisis de la metodología actualmente vigente en el entorno empresarial.

\subsection{Anotación de preguntas}
Se ha llevado a cabo una elicitación de requisitos mediante la implementación de un cuestionario electrónico dirigido al personal técnico de la sede de Zuatzu en LKS Next, incluyendo coordinadores de equipo con experiencia en la incorporación de desarrolladores junior. En dicho cuestionario, se solicitó a los profesionales que, acorde a su ámbito de especialización, anotasen las interrogantes que formularían a un sistema de onboarding hipotético.

Se recabaron un total de 63 interrogantes aportadas por 8 profesionales del ámbito del desarrollo software. Posteriormente, dichas cuestiones fueron ampliadas y categorizadas mediante la utilización del modelo Claude 3.7 Sonnet\footnote{Claude 3.7 Sonnet: \url{https://www.anthropic.com/claude/sonnet}} en su versión razonadora, tras lo cual se procedió a una anotación manual para eliminar elementos redundantes. La compilación clasificada final puede consultarse en el Anexo \ref{listado:preguntas}, donde se han identificado los siguientes dominios de conocimiento:

\begin{itemize}
\item\textbf{Información general: }objetivo, finalidad y contexto del proyecto.
\item\textbf{Entorno y despliegue: }entornos de desarrollo y sistemas de despliegue disponibles.
\item\textbf{Gestión del proyecto: }comunicación y coordinación del equipo, metodología de contribución y gestión de tareas y requisitos.
\item\textbf{Estándares y prácticas: }estándares de codificación, estándares visuales, estándares de integración y aspectos legales.
\item\textbf{Documentación: }información sobre las fuentes de documentación disponibles para el proyecto, incluyendo su ubicación.
\item\textbf{Recursos adicionales: }recursos externos al proyecto como formaciones y documentación de librerías.
\item\textbf{Arquitectura del sistema: }cuestiones de diseño del código fuente sobre todos los niveles de abstracción del modelo C4\cite{brown_c4_2018}. Este modelo define los niveles de abstracción de un proyecto software en 4 niveles:
\begin{itemize}
\item Actores y sistemas externos con los que interactúa el sistema
\item Contenedores y aplicaciones que componen el sistema
\item Componentes que definen cada contenedor
\item Diagrama de clases e interfaces de cada componente
\end{itemize}
\end{itemize}

A modo de ejemplo, la siguiente pregunta se ha clasificado como parte de la gestión del proyecto: \texttt{¿Dónde puedo encontrar la descripción detallada de las tareas asignadas o disponibles?}

\subsection{Metodología empresarial}
La metodología implementada ha sido consultada directamente con los desarrolladores de LKS Next. Al tratarse de una empresa consultora, las prácticas aplicadas presentan variaciones en función del cliente específico al que se presta servicio. No obstante, se ha identificado que dicha metodología se fundamenta en las siguientes directrices comunes:
\begin{itemize}
\item\textbf{Gestión del tiempo: }los integrantes del equipo utilizan una aplicación propietaria para el registro y contabilización de las horas invertidas en cada proyecto.
\item\textbf{Gestión de tareas: }la asignación, seguimiento y monitorización de tareas se efectúa a través de herramientas especializadas como Jira\footnote{Jira: \url{https://www.atlassian.com/es/software/jira}}.
\item\textbf{Documentación: }la documentación técnica y funcional de los proyectos se distribuye en diversas plataformas, incluyendo páginas web específicas o directorios designados dentro del propio repositorio del proyecto.
\item\textbf{Repositorio del proyecto: }la gestión del código fuente se realiza de manera centralizada en un repositorio común, alojado en plataformas como GitHub o el GitLab\footnote{GitLab: \url{https://about.gitlab.com/}} propio de LKS Next.

\end{itemize}
Adicionalmente, el flujo de trabajo establecido entre los equipos de diseño y desarrollo sigue una metodología específica. En una entrevista con un miembro del departamento de diseño (acta disponible en Anexo \ref{anexo:acta_diseñadores}), se ha documentado el siguiente proceso:

En primera instancia, los requisitos visuales del proyecto se documentan en un repositorio de Confluence\footnote{Confluence: \url{https://www.atlassian.com/es/software/confluence}} sincronizado con el cliente. Posteriormente, el equipo de diseño elabora un prototipo visual empleando herramientas especializadas como Figma\footnote{Figma: \url{https://www.figma.com/}}. Una vez aprobados los diseños, estos se transforman en maquetas HTML funcionales, las cuales son compartidas con el equipo de desarrollo a través del sistema de almacenamiento en la nube de Google Drive.


\section{Recursos a utilizar}
Tras definir los requisitos fundamentales y los dominios de conocimiento necesarios, se requiere establecer el contexto específico donde integrar el sistema a desarrollar. 
\subsection{Proyecto software}

Debido a las restricciones legales existentes, no es viable la utilización de proyectos comerciales en entornos de producción. Bajo estas restricciones, para el desarrollo del presente trabajo se ha seleccionado el proyecto propietario IA-core-tools. Este constituye un conjunto de herramientas para el desarrollo de agentes basados en LLM con capacidad de acceso a repositorios de información mediante RAG. La aplicación está implementada en Flask\footnote{Flask: \url{https://flask.palletsprojects.com/en/stable/}} y proporciona una API que facilita el desarrollo de agentes de manera interactiva a través de una interfaz visual, permitiendo a los desarrolladores de LKS Next crear soluciones para proyectos en producción.

Se dispone de acceso completo al repositorio de GitLab del proyecto, lo que proporciona información detallada sobre todos los usuarios, contribuciones realizadas, gestión de incidencias y administración de tareas. Asimismo, se dispone de las maquetas HTML utilizadas para crear la interfaz de usuario. 

Cabe destacar que la metodología de gestión implementada difiere ligeramente de la empleada en proyectos comerciales. En este caso, el desarrollador principal es Aritz Galdos (supervisor en la empresa del presente proyecto), quien coordina reuniones tanto presenciales como virtuales con los contribuyentes del proyecto. Debido a las restricciones legales mencionadas anteriormente, no se dispone de acceso a la aplicación de gestión horaria del proyecto.

\subsection{Recursos generados}
Con el objetivo de una evaluación detallada del sistema, se ha procedido a la generación de documentación que replica con la mayor fidelidad posible las características del proyecto original. Para este propósito, se ha empleado nuevamente el modelo Claude 3.7 Sonnet, utilizando como datos de entrada toda la información de gestión previamente mencionada, así como ficheros clave extraídos del repositorio del código fuente. Posteriormente, se ha realizado un proceso de refinamiento manual de la documentación generada con el objetivo de eliminar imprecisiones e información errónea.

En una primera fase, se han elaborado tres documentos específicos relacionados con la sección visual del proyecto: \opus{funcionamiento_y_diseno_interfaz}, \opus{guia_de_estilos_visual} y \opus{limitaciones_y_mejoras_pendientes}. Estos documentos han sido incorporados a un repositorio de Confluence para simular la documentación habitualmente proporcionada por los diseñadores del equipo.

Seguidamente, se han desarrollado trece documentos que constituyen una simulación de la ``documentación oficial'' del proyecto. Estos documentos recogen los elementos necesarios para la comprensión de la gestión, diseño y metodología de contribución del proyecto: \opus{arquitectura-software.md}, \opus{despliegue.md}, \opus{equipo-y-comunicacion.md}, \opus{estandares-codigo.md}, \opus{flujos-trabajo.md}, \opus{guia-contribucion.md}, \opus{informacion-cliente.md}, \opus{metodologia.md}, \opus{modelo-negocio.md}, \opus{onboarding.md}, \opus{README.md}, \opus{referencias-tecnicas.md} y \opus{sistema-gestion-tareas.md}.

Por último, se ha generado la ``documentación API'' del proyecto mediante la utilización del agente RepoAgent\cite{luo_repoagent_2024}. Esta documentación contiene explicaciones detalladas de todas las clases y métodos implementados en Python, que conforman la estructura  del dominio y del modelo de negocio. El uso de este agente externo se explica detalladamente en la sección \ref{sec:herramientas_codigo}.

Esta infraestructura documental permite simular el proceso de onboarding de un nuevo desarrollador en el proyecto IA-core-tools, evaluando las respuestas del sistema agéntico ante las interrogantes típicas identificadas en los dominios de conocimiento.

Una vez definidos los requisitos y recursos del sistema, el siguiente capítulo comienza con la descripción de la implementación.







