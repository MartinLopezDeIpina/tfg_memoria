Resumen de diferentes partes: 
- Librerías utilizadas?? -> LangChain, LangGraph, PgVector

- Ajuste de agentes -> LoRa? -> comentar lo de los agentes instruct.
- El protocolo MCP
- El estado del arte en arquitecturas de agentes LLM y sistemas RAG
- 
- El estado del arte en agentes integrados a proyectos software + contexto de onboarding ~quizás en la introducción.
- Redes neuronales?

2. Agentes LLM (Fundamentos y funcionamiento básico)

¿Qué son los agentes LLM?
2.1. Modelos LLM
2.2. Interacción con herramientas
2.3. Abstracciones en frameworks


\section{Agentes LLM}

Los agentes de Inteligencia Artificial son programas informáticos que implementan modelos computacionales para ejecutar diversas funciones específicas del contexto en el que se aplican. Tras siete décadas y media de investigación, los esfuerzos en el campo se han focalizado en agentes basados en Grandes Modelos de Lenguaje (LLM). 

\subsection{Modelos LLM}

Los LLM son redes neuronales especializadas en el procesamiento del lenguaje natural, basados en la arquitectura Transformer. Esta arquitectura se fundamenta en el mecanismo de atención, el cual transforma la representación del texto para incorporar información contextual de manera escalable al hardware. 

Para comprender el funcionamiento de estos agentes, resulta imprescindible asimilar previamente conceptos como la tokenización y las representaciones vectoriales del lenguaje.

\paragraph{Tokens}
Los tokens constituyen la unidad mínima de texto que el modelo puede procesar. Dado que dichos modelos operan sobre estructuras matemáticas, requieren transformar el lenguaje natural en representaciones matriciales. Para lograr esta conversión, el texto se segmenta en dichas unidades mínimas, que pueden corresponder a caracteres individuales, fragmentos de texto o palabras completas. El conjunto íntegro de estas unidades reconocibles por el modelo configura su vocabulario. 

\paragraph{Representaciones vectoriales}
Constituyen vectores numéricos de dimensionalidad fija que codifican la semántica inherente a cada token. Estos vectores pueden comprender desde 768 dimensiones en arquitecturas como BERT-base hasta superar las 16.000 dimensiones en los modelos más avanzados del estado del arte. Por ejemplo, una dimensión específica podría especializarse en representar conceptos abstractos. En este contexto, la representación vectorial del token ``animal`` contendría un valor más elevado en dicha dimensión que la correspondiente al término ``gato``, reflejando su mayor grado de abstracción conceptual.


\subsection{Interacción con herramientas externas}
Los agentes LLM poseen la capacidad de interactuar con diversas herramientas como búsquedas web\cite{nakano_webgpt_nodate}, bases de datos o interfaces de usuario. Fundamentalmente, un LLM solo genera tokens de texto, por lo que la integración de herramientas se implementa mediante palabras clave que el modelo puede incluir en su salida. Para ello, en el texto de entrada se especifica el esquema de la función a utilizar y, si decide emplearla, el modelo generará el texto correspondiente. Posteriormente, se procesa la respuesta para extraer llamadas a funciones si las hubiese.

La interacción con herramientas es típicamente alternante. Tras realizar la llamada a la herramienta, la salida de esta se utilizará como entrada para el siguiente mensaje del modelo. La figura \ref{fig:herramientas} ilustra el esquema de un agente con acceso a una API del clima. Como el modelo carece de información climática en tiempo real, se le indica en el prompt la posibilidad de invocar esta función. Al incluir la llamada en su texto de salida, se ejecuta la función y su respuesta se transmite al modelo para generar el resultado final.

%xu_rewoo_2023 -> rewoo para no pasar observaciones a react ~orquestar
%react todo
%wang_executable_2024 -> tools con código, igual mejor donde herramientas?


\begin{figure}[H]
  \centering
  \includegraphics[width=1\linewidth]{figures/herramienta.png}
  \caption{Ejemplo de interacción de un modelo LLM con una herramienta externa.}
  \label{fig:herramientas}
\end{figure}

\subsection{Abstraciones en frameworks}
Aunque los agentes LLM son una tecnología de reciente surgimiento, ya se han desarrollado frameworks que estandarizan su implementación. Estas estructuras de trabajo ofrecen abstracciones de alto nivel para reutilizar funcionalidades comunes presentes en la mayoría de sistemas de agentes.
Las funcionalidades principales que estos frameworks proporcionan son:
\begin{itemize}
\item {\textbf{Gestión de modelos:}} La ejecución de LLM requiere dominio del modelo empleado, ya que cada uno posee tokenizadores específicos y esquemas propios de entrada/salida. Los frameworks ofrecen interfaces unificadas, facilitando el uso de diversos modelos sin conocimientos técnicos precisos.
\item {\textbf{Interacción conversacional:}} La comunicación con los modelos se efectúa mediante un esquema conversacional, donde el modelo recibe un texto de entrada y genera una respuesta correspondiente. Las respuestas y entradas se concatenan secuencialmente para preservar el contexto de la conversación, cada consulta subsiguiente incorpora todos los intercambios precedentes.
  \item {\textbf{Uso de herramientas externas:}} El desarrollador únicamente debe especificar la función que desea incorporar, toda la compljeidad de la interacción se abstrae en el framework.
\item {\textbf{Interacción entre agentes:}} Los agentes pueden establecer comunicación entre sí, permitiendo la construcción de sistemas con mayor complejidad. Algunos frameworks establecen protocolos que definen las modalidades de comunicación entre los distintos agentes.
\end{itemize}

Entre las más populares se encuentran LangChain, LangGraph, LlamaIndex, AutoGen, CrewAI, Smolagents y el reciente OpenAI Agents. 

\section{Model Context Protocol}
todo: referencias a las docs.

El Model Context Protocol (MCP), desarrollado por Anthropic, estandariza la comunicación entre agentes LLM y herramientas. Permite que aplicaciones diversas ofrezcan herramientas a agentes externos sin exponer detalles de implementación. Comparable al modelo OSI, el MCP opera en un nivel de abstracción inferior a los frameworks, proporcionando una capa de interoperabilidad.

La figura \ref{fig:mcp} ilustra el esquema operativo del protocolo. En este contexto, los desarrolladores de Jira y GitHub han implementado un servidor MCP que proporciona las herramientas disponibles al cliente MCP. Este servidor realiza la traducción de las interacciones necesarias con las API dichas aplicaciones, permitiendo que el agente LLM únicamente requiera conocer el esquema funcional que debe utilizar. Por su parte, el cliente MCP se encarga de administrar la comunicación con los diferentes servidores, facilitando que el agente acceda directamente a las herramientas disponibles.

\begin{figure}[H]
  \centering
  \includegraphics[width=1\linewidth]{figures/mcp.png}
  \caption{Esquema de funcionamiento del Model Context Protocol.}
  \label{fig:mcp}
\end{figure}


El protocolo ofrece dos modos de operación para establecer la comunicación entre cliente y servidor:
\begin{itemize}
  \item{\textbf{Comunicación SSE: } El protocolo Server-Sent Events (SSE) establece un canal de comunicación unidireccional sobre HTTP desde el servidor hacia el cliente. Proporciona actualizaciones en tiempo real con capacidad de streaming. En el protocolo MCP, el cliente efectúa solicitudes para la ejecución de herramientas en el servidor mediante HTTP, a lo que el servidor puede responder mediante eventos SSE.}
\item{\textbf{Comunicación STDIO: } El protocolo de entrada y salida estándar (STDIO) facilita la comunicación bidireccional entre cliente y servidor a nivel de proceso en el sistema operativo. Este mecanismo permite el intercambio de información en formato JSON a través de los canales estándar del sistema. Su diseño, orientado principalmente a entornos locales, restringe la conexión a un único cliente por servidor al limitarse a la comunicación entre dos procesos.}
\end{itemize}
La aplicación de escritorio claude-desktop de Anthropic constituye un reflejo del potencial del protocolo. Esta plataforma ofrece la posibilidad de interactuar con servidores preconfigurados mediante una configuración mínima. Implementando el protocolo STDIO, la aplicación ejecuta los servidores distribuidos por terceros a través de gestores de paquetes como UV, o alternativamente con Docker. Al incorporar un cliente MCP en la aplicación, consigue integrar las herramientas disponibles en la interfaz de chat con los modelos de Anthropic.

\section{Estado del arte en arquitecturas de agentes LLM}

La comunidad científica ha creado arquitecturas de agentes para optimizar el rendimiento de los modelos disponibles. La arquitectura RAG se distingue por complementar la entrada del modelo con información recuperada de documentos relevantes. Otras propuestas se centran en mejorar la comunicación y coordinación entre agentes.

\subsection{Arquitectura RAG}

Los modelos LLM poseen un conocimiento restringido a los datos con los que fueron entrenados. Para superar esta limitación, la arquitectura RAG (Retrieval-Augmented Generation) complementa la generación del LLM mediante la recuperación de información relevante desde repositorios de conocimiento externos. La figura \ref{fig:rag} ilustra un ejemplo de su funcionamiento.    

\begin{figure}[H]
  \centering
  \includegraphics[width=0.5\linewidth]{figures/RAG.png}
  \caption{Esquema de funcionamiento de la arquitectura RAG en un LLM \href{https://www.clarifai.com/blog/what-is-rag-retrieval-augmented-generation}{Fuente}.}
  \label{fig:rag}
\end{figure}

La recuperación de documentos relevantes se puede implementar mediate recuperadores dispersos: expresiones regulares, búsqueda de n-gramas, palabras clave, entre otras. No obstante, el enfoque predominante consiste en el uso de recuperadores densos, conocidos como indexación vectorial. En este método, los documentos se transforman en vectores, generalmente mediante LLMs especializados en codificación, denominados Embedders. Al representar los documentos en un espacio vectorial, es posible recuperar aquellos semánticamente más pertinentes mediante la comparación del vector de consulta con los vectores de los documentos indexados, utilizando métricas como la distancia coseno.

\subsubsection{Estrategias RAG avanzadas}
La optimización del rendimiento en arquitecturas RAG ha sido ampliamente estudiada \cite{zhu_retrieving_2021}\cite{gao_retrieval-augmented_2024}, enfocándose en tres áreas principales: el procesado de documentos, los sistemas de recuperación y la mejora del flujo de generación.
\begin{itemize}
  \item {\textbf{Procesado de documentos:}} La calidad de la indexación documental determina la eficacia del sistema. Entre las estrategias destacadas figuran la eliminación de ruido textual, el ajuste del tamaño óptimo de los segmentos indexados, y la técnica de ventana solapada. Esta última superpone fragmentos para preservar los datos situados en las intersecciones de los segmentos. 

\item {\textbf{Sistemas de recuperación:}} Los recuperadores densos basados en representaciones, ilustrados en la figura \ref{fig:rag}, precalculan las representaciones vectoriales de los documentos mediante una indexación independiente de las consultas. Por otro lado, los recuperadores basados en interacción, proponen calcular el vector para cada documento junto al vector de consulta durante el tiempo de ejecución, obteniendo así una representación más rica que captura las relaciones contextuales específicas entre ambos elementos\cite{ma_query_nodate}\cite{levine_standing_2022}.

Para contrarrestar el sobrecoste computacional de este enfoque, se han desarrollado técnicas de indexación híbrida que combinan ambos métodos. Estas técnicas permiten realizar una búsqueda inicial utilizando representaciones vectoriales y posteriormente refinar los resultados mediante la extracción interactiva\cite{khattab_relevance-guided_2021}. 
  
\item {\textbf{Optimización del flujo de generación:}} Para tareas que requieren reflexión documentada, se han desarrollado sistemas de salto múltiple que alternan entre recuperación y generación\cite{khattab_demonstrate-search-predict_2023}\cite{shao_enhancing_2023}\cite{qi_answering_2021}\cite{zheng_take_2024}\cite{trivedi_interleaving_2023}. Estos flujos permiten al sistema examinar críticamente la información inicialmente recuperada, identificar lagunas informativas, y formular las consultas correspondientes para subsanar dichas carencias.

\end{itemize}

%lazaridou_internet-augmented_2022 -> rag con internet
%cheng_uprise_2023 -> rag sobre prompts

%li_structure-aware_2023 -> datos estructurados aprender a lenguaje natural


%kang_knowledge_2023 -> combinar RAG con GNN para relaciones 

\subsection{Arquitecturas de interacción entre agentes}
La interacción entre agentes LLM constituye un campo de investigación activo, distinguiéndose diversos avances en módulos de memoria, planificación e interacción multiagente\cite{wang_survey_2024}.
\begin{itemize}
  \item{\textbf{Módulos de memoria}}: En una interacción conversacional, el modelo procesa todos los mensajes previos, pudiendo generar un contexto excesivamente amplio. Para mitigar este problema, se han desarrollados módulos de memoria que almacenan información relevante de interacciones pasadas de forma resumida\cite{zhang_building_2024}\cite{fischer_reflective_2023}\cite{liang_unleashing_2023}. Esta memoria puede consultarse posteriormente mediante RAG, permitiendo recuperar los elementos más relevantes según el contexto\cite{zhao_expel_2024}.

Algunos módulos se inspiran en la estructura de la memoria humana\cite{zhong_memorybank_2024}, incorporando mecanismos que almacenan información con distintas temporalidades y niveles de relevancia\cite{wang_survey_2024}\cite{park_generative_2023}.


\item{\textbf{Planificación}}: Los mecanismos de planificación potencian el razonamiento de los agentes sobre sus acciones futuras.

Entre estos mecanismos destaca el prompting de cadena de pensamiento (Chain of Thought)\cite{wei_chain--thought_2022}\cite{kojima_large_2022}, que instruye al modelo para elaborar un razonamiento secuencial previo a su decisión final, permitiendo así descomponer problemas complejos paso a paso.
Partiendo de este enfoque, estrategias avanzadas como la autoconsciencia\cite{liang_unleashing_2023} lo amplían mediante la generación de múltiples cadenas de razonamiento independientes y la posterior selección de la respuesta óptima entre ellas\cite{yao_tree_nodate}\cite{wang_recmind_2024}.

De manera complementaria, las técnicas de reflexión\cite{shinn_reflexion_nodate}\cite{shinn_reflexion_nodate}\cite{madaan_self-refine_nodate}\cite{miao_selfcheck_2023} implementan un proceso iterativo donde el propio modelo evalúa y refina sus respuestas.

Por otro lado, la estructuración de acciones constituye una metodología ampliamente adoptada en la planificación\cite{lin_swiftsage_nodate}\cite{huang_language_nodate}\cite{wang_describe_2024}. Esta técnica permite definir planes de alto nivel que posteriormente se desglosan en acciones específicas ejecutables por los agentes\cite{zhu_ghost_2023}\cite{song_llm-planner_2023}\cite{wang_voyager_2023}\cite{liu_odyssey_2024}. El análisis de las interdependencias entre estas acciones permite verificar la validez de los planes generados\cite{raman_planning_nodate}\cite{liu_llmp_2023}\cite{dagan_dynamic_2023}.

Los modelos razonadores como o1 o DeepSeek-R1 incorporan estas técnicas de forma nativa\cite{noauthor_deepseek-r1deepseek_r1pdf_nodate}. Son entrenados con datos que incluye ejemplos de razonamiento y planificación, lo que les permite generar respuestas más precisas y coherentes.


\item{\textbf{Interacción entre agentes: }}Los sistemas multi-agente implementan una arquitectura donde diversos agentes especializados son coordinados por un componente central denominado orquestador\cite{karpas_mrkl_2022}\cite{ge_openagi_nodate}. En este paradigma, cada agente se especializa en una función particular, como la búsqueda de información, la ejecución de herramientas o la generación de texto. El orquestador evalúa las consultas entrantes y las dirige hacia el agente más competente para resolverlas.

Enfoques complementarios proponen la interacción directa entre agentes especializados como mecanismo de retroalimentación\cite{zhuge_mindstorms_2023}\cite{du_improving_nodate}. Por ejemplo, ChatDev\cite{qian_chatdev_2024} establece un sistema de colaboración entre agentes programadores, testers y gestores para abordar problemas de ingeniería de software. MetaGPT\cite{hong_metagpt_2024} refina esta propuesta al implementar un protocolo de comunicación basado en el patrón publicador/suscriptor entre los agentes, permitiéndoles difundir información de forma selectiva. 
\end{itemize}

\subsection{Agentes LLM en proyectos de software}

La integración de agentes LLM en proyectos software ha sido objeto de investigación en diversas áreas. 

\subsection{Ajuste de modelos para agentes LLM}
Los llm del estado del arte suelen venir con la capacidad de ajustada de llamar tools, algunos estudios demuestran que modelos pequeños pueden sobrepasar a modelos grandes en tareas específicas si se entrenan para ello. 

%todo en capitulo agente código:
%rana_sayplan_2023 -> robot con path de la casa -> similar a repo
%gramopadhye_generating_2023 -> mejora al anterior con few shots entorno



















