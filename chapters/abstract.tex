El proceso de incorporación de nuevos desarrolladores a proyectos software presenta desafíos debido a la sobrecarga informativa que supone sintetizar información dispersa en múltiples fuentes como código fuente, documentación y sistemas de gestión. Este trabajo implementa y evalúa un sistema multiagente basado en grandes modelos de lenguaje (LLM) para optimizar el proceso de onboarding, compuesto por cinco agentes especializados con acceso a fuentes específicas mediante técnicas RAG, protocolo MCP y mecanismos de orquestación con planificadores y coordinadores. El sistema incorpora estrategias de prompting, memoria persistente y diseño adaptativo, evaluándose mediante métricas automatizadas sobre un dataset de 46 preguntas derivadas de una elicitación de requisitos con profesionales de LKS Next. Los resultados demuestran una precisión superior al 80\% en consultas de onboarding, destacando la efectividad de las arquitecturas con planificación unificada y la eficacia del protocolo MCP para la integración de sistemas externos. Las evidencias obtenidas posicionan a los agentes LLM como una solución prometedora para automatizar el onboarding en entornos productivos bajo supervisión apropiada.

