El proceso de incorporación de nuevos desarrolladores a proyectos software presenta desafíos debido a la sobrecarga informativa que supone sintetizar información dispersa en múltiples fuentes como código fuente, documentación y sistemas de gestión. Este trabajo implementa y evalúa un sistema multi-agente basado en Grandes Modelos de Lenguaje (LLM) para automatizar y optimizar el proceso de onboarding, compuesto por cinco agentes especializados con acceso a fuentes específicas mediante técnicas RAG, protocolo MCP y mecanismos de orquestación con planificadores y coordinadores. El sistema incorpora técnicas de prompting, memoria persistente y diseño adaptativo, evaluándose mediante métricas automatizadas sobre un dataset de 46 preguntas derivadas de una elicitación de requisitos con profesionales de LKS Next. Los resultados indican una precisión superior al 80\% en consultas de onboarding, destacando la eficacia del protocolo MCP para la integración de sistemas externos y las arquitecturas con planificación unificada. Los datos obtenidos posicionan a los agentes LLM como una solución prometedora para automatizar el onboarding en entornos productivos bajo supervisión humana adecuada.

