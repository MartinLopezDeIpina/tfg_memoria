Este proyecto se relaciona principalmente con el ODS 4 (Educación de Calidad), proporcionando un sistema de formación más asequible, y contribuye indirectamente al ODS 10 (Reducción de las Desigualdades) y al ODS 8 (Trabajo Decente y Crecimiento Económico).

Los sistemas de incorporación tradicionales dependen de la disponibilidad de mentores humanos, cuya asistencia consume recursos valiosos que no siempre es viable invertir, complicando estos procesos de integración. Mediante la exploración realizada en este proyecto, se contribuye al desarrollo de sistemas de incorporación automatizados que no requieren tales recursos humanos especializados. De este modo, la calidad de la formación en proyectos software se independiza parcialmente de los recursos disponibles, proporcionando un marco educativo más igualitario alineado con el ODS 4 y reduciendo potencialmente las desigualdades de acceso a la formación técnica especializada conforme al ODS 10.

No obstante, al tratarse los agentes LLM de tecnologías emergentes, los sistemas en producción son escasos y su disponibilidad se limita a grupos con mayores recursos económicos. Por ejemplo, el sistema agéntico Claude Code está disponible únicamente mediante suscripción mensual superior a 100 euros en España, mientras que exploraciones que requieren múltiples instancias simultáneas pueden superar los 1000 euros mensuales.

Paradójicamente, si estas tecnologías resultan tan eficaces como prometen, las organizaciones con mayores recursos serán las más beneficiadas, disponiendo de las versiones más avanzadas y obteniendo ventajas competitivas significativas. En un mercado donde todas las organizaciones compiten, esto podría incrementar las desigualdades.

La solución más directa residiría en regular estos sistemas para garantizar una competencia equitativa. Sin embargo, tales regulaciones limitarían a organizaciones como OpenAI, cuya labor consiste precisamente en desarrollar estas tecnologías. Cabe preguntarse si habrían sido desarrolladas inicialmente bajo un marco regulatorio más restrictivo. ¿Se debe limitar el avance tecnológico en favor de la igualdad? Podría argumentarse que la calidad de vida del más próspero en la época medieval era inferior a la del más desfavorecido en la actualidad.

Por tanto, ¿fomentan estos sistemas el crecimiento económico y la reducción de las desigualdades? Probablemente sí, en caso de ser regulados. ¿Deberían serlo? Afortunadamente, esta cuestión política recae en manos de aquellos cuya labor investigadora se centra en este tipo de dilemas.

