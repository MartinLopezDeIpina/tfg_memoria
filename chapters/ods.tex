Este proyecto se relaciona principalmente con el ODS 4 (Educación de Calidad), facilitando formación más asequible, y contribuye al ODS 10 (Reducción de las Desigualdades).

Los sistemas de incorporación tradicionales dependen de mentores humanos. Esta asistencia consume recursos valiosos que no siempre es viable invertir, complicando los procesos de integración. Mediante la exploración realizada en este proyecto, se contribuye al desarrollo de sistemas de incorporación automatizados que no requieren tales recursos humanos especializados. De este modo, la calidad de la formación en proyectos de software se independiza parcialmente de los recursos disponibles, proporcionando un marco educativo más igualitario alineado con el ODS 4 y reduciendo las desigualdades de acceso a la formación técnica cualificada conforme al ODS 10.

No obstante, al tratarse los agentes LLM de tecnologías emergentes, los sistemas en producción son escasos y su disponibilidad se limita a grupos con mayores recursos económicos. Por ejemplo, el sistema agéntico Claude Code está disponible únicamente mediante una suscripción mensual superior a 100 euros en España, mientras que exploraciones que requieren múltiples instancias simultáneas pueden superar los 1000 euros mensuales.

Paradójicamente, si estas tecnologías resultan tan eficaces como prometen, las organizaciones con mayores recursos serán las más beneficiadas, disponiendo de las versiones más avanzadas y obteniendo como resultado ventajas competitivas. En un mercado donde todas las organizaciones rivalizan, esto podría incrementar las desigualdades.

La solución más directa residiría en regular estos sistemas para garantizar una competencia equitativa. Sin embargo, tales regulaciones limitarían a organizaciones privilegiadas como OpenAI, cuya labor consiste precisamente en desarrollar estas tecnologías. Cabe preguntarse si habrían sido desarrolladas inicialmente bajo un marco regulatorio más restrictivo. Esto plantea la cuestión: ¿se debe limitar el avance tecnológico en favor de la igualdad? Podría argumentarse que la calidad de vida del más próspero en la época medieval era inferior a la del más desfavorecido en la actualidad.

Por tanto, ¿fomentan estos sistemas la educación de calidad y la reducción de las desigualdades? En caso de ser regulados, probablemente sí. ¿Deberían serlo? Posicionarse ante esta cuestión política no sería más que una especulación sesgada por consideraciones personales. Afortunadamente, es responsabilidad de aquellos cuya labor investigadora se centra en este tipo de dilemas, y cuyos esfuerzos han resultado en los ODS.
