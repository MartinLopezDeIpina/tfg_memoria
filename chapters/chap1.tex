En la última década, el campo de la generación de contenido mediante inteligencia artificial ha experimentado avances significativos \cite{vaswani_attention_2023}, dando lugar a modelos computacionales denominados grandes modelos de lenguaje (LLM). La capacidad de comprensión y generación de texto en lenguaje natural de estos modelos abre numerosas posibilidades, permitiendo la automatización de tareas cada vez más complejas. El uso programático de dichos sistemas a través de agentes LLM se ha convertido en objeto de estudio en múltiples áreas de aplicación \cite{noauthor_design_nodate, singh_exploring_2023, dong_self-collaboration_2024}.

Dentro de estos casos de uso, la incorporación de nuevos desarrolladores a proyectos software (\textit{Onboarding}) resulta especialmente relevante. Este proceso requiere la búsqueda y síntesis de grandes cantidades de información en múltiples fuentes como código fuente, documentación y aplicaciones de gestión. Aunque muchos proyectos buscan automatizar el proceso de desarrollo completo \cite{qian_chatdev_2024, acharya_devin_2025, noauthor_aider-aiaider_2025}, pocos trabajos han explorado específicamente el potencial formativo en este contexto.

Para optimizar la eficacia de dichos agentes, han surgido diversas arquitecturas como \textit{Retrieval Augmented Generation} (RAG) \cite{gao_retrieval-augmented_2024} y \textit{Reasoning and Acting} (ReAct) \cite{yao_react_2023}. Asimismo, tecnologías como LangChain\footnote{LangChain: \url{https://www.langchain.com/}}, LangGraph\footnote{LangGraph: \url{https://www.langchain.com/langgraph}} y LangSmith\footnote{LangSmith: \url{https://www.langchain.com/langsmith}} constituyen marcos de desarrollo para facilitar su aplicación. En este contexto, el \textit{Model Context Protocol} (MCP)\footnote{Model Context Protocol: \url{https://docs.anthropic.com/en/docs/agents-and-tools/mcp}} emerge con la idea de estandarizar el funcionamiento de los agentes independientemente de su implementación específica.


Aprovechando este auge tecnológico, LKS Next\footnote{LKS NEXT: \url{https://www.lksnext.com/es/}} desarrolla productos que automatizan tareas en producción mediante estos sistemas. No obstante, el carácter innovador requiere un estudio constante de los últimos avances, objetivo que persigue el presente Trabajo de Fin de Grado. Mediante la exploración de estas tecnologías en el ámbito del Onboarding, se busca investigar arquitecturas de aplicación útiles y evaluar el uso del protocolo MCP.

El resto del presente documento se organiza como sigue: el Capítulo \ref{ch:chap2} introduce en detalle tanto el dominio de aplicación como técnico. Los Capítulos \ref{ch:chap3} y \ref{ch:chap4} presentan la planificación del proyecto y los requisitos correspondientes. Posteriormente, los Capítulos \ref{ch:chap5}, \ref{ch:chap6}, \ref{ch:chap7}, \ref{ch:chap8} y \ref{ch:chap9} detallan el desarrollo del proyecto y el sistema implementado, comenzando por el diseño y arquitectura general, para continuar con los agentes concretos y su evaluación. Finalmente, el Capítulo \ref{ch:chap10} presenta el seguimiento y control del proyecto, mientras que los Capítulos \ref{ch:chap11} y \ref{ch:chap12} exponen los retos y conclusiones finales.
