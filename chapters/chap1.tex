En la última década, el campo de la generación de contenido mediante inteligencia artificial ha experimentado avances significativos \cite{vaswani_attention_2017}, culminando en el desarrollo de los grandes modelos de lenguaje (LLM). Estos modelos, capaces de comprender y generar texto en lenguaje natural, han abierto numerosas posibilidades para la automatización de tareas cada vez más complejas. Su uso programático, representado mediante agentes LLM, se ha convertido en objeto de estudio en múltiples áreas de aplicación \cite{xiaoliang_design_2024, dong_self-collaboration_2024}.

Dentro de estos casos de uso, la incorporación de nuevos desarrolladores a proyectos software (\textit{onboarding}) resulta especialmente relevante. Este proceso implica la búsqueda y síntesis de información dispersa en código fuente, documentación y aplicaciones de gestión, tareas que los agentes LLM abordan con especial eficacia. Aunque muchos proyectos buscan automatizar el proceso de desarrollo completo \cite{qian_chatdev_2024, acharya_devin_2025}, pocos trabajos han explorado específicamente el potencial formativo en este contexto.

Paralelamente, para optimizar la efectividad de dichos agentes, han surgido diversas arquitecturas como \textit{Retrieval Augmented Generation} (RAG) \cite{gao_retrieval-augmented_2024} y \textit{Reasoning and Acting} (ReAct) \cite{yao_react_2023}. Asimismo, tecnologías como LangChain\footnote{LangChain: \url{https://www.langchain.com/}}, LangGraph\footnote{LangGraph: \url{https://www.langchain.com/langgraph}} y LangSmith\footnote{LangSmith: \url{https://www.langchain.com/langsmith}} constituyen marcos de desarrollo para facilitar su aplicación. En este contexto, el \textit{Model Context Protocol} (MCP)\footnote{Model Context Protocol: \url{https://docs.anthropic.com/en/docs/agents-and-tools/mcp}} emerge con el objetivo de estandarizar el funcionamiento de los agentes independientemente de su implementación específica.


Aprovechando este auge tecnológico, LKS Next\footnote{LKS Next \url{https://www.lksnext.com/es/}} desarrolla productos que automatizan tareas en producción mediante estos sistemas. Sin embargo, el carácter innovador requiere un estudio constante de los últimos avances, objetivo que persigue el presente Trabajo de Fin de Grado. Mediante la exploración de estas tecnologías en el ámbito del onboarding, se busca analizar arquitecturas de aplicación efectivas y evaluar el uso del protocolo MCP.

El sistema desarrollado, junto con su código fuente completo, se encuentra disponible públicamente en el repositorio del proyecto en GitHub: \url{https://github.com/MartinLopezDeIpina/tfg\_agentes\_software}

El resto del presente documento se organiza como sigue: el Capítulo \ref{ch:chap2} introduce tanto el dominio de aplicación como el técnico. Los Capítulos \ref{ch:chap3} y \ref{ch:chap4} presentan la planificación del proyecto y los requisitos correspondientes. Posteriormente, los Capítulos \ref{ch:chap5}, \ref{ch:chap6}, \ref{ch:chap7}, \ref{ch:chap8} y \ref{ch:chap9} detallan el desarrollo del proyecto y el sistema implementado, comenzando por el diseño y arquitectura general, para continuar con los agentes concretos y su evaluación. Finalmente, el Capítulo \ref{ch:chap10} presenta el seguimiento y control del proyecto, mientras que los Capítulos \ref{ch:chap11} y \ref{ch:chap12} exponen los retos y conclusiones finales.
