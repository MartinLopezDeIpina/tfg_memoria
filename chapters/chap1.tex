En la última década, el campo de la generación de contenido mediante inteligencia artificial ha experimentado avances significativos, resultando en modelos computacionales denominados grandes modelos de lenguaje (LLM). La capacidad de comprensión y generación de texto en lenguaje natural de estos modelos abre un abanico de posibilidades inimaginable, permitiendo la automatización de tareas cada vez más complejas. El uso programático de dichos sistemas a través de agentes LLM está siendo objeto de estudio en múltiples áreas de aplicación.

Entre todos estos casos de uso, la incorporación de nuevos desarrolladores a proyectos software (\textit{Onboarding}) constituye un área con gran potencial exploratorio. El proceso de aprendizaje necesario requiere búsqueda y síntesis de extensas cantidades de información en múltiples fuentes: código fuente, documentación, aplicaciones de gestión, etc. Aunque muchos proyectos tienen la ambición de automatizar el procedimiento de desarrollo completo \cite{}, pocos trabajos han explorado el potencial formativo en este contexto.

Para optimizar la eficacia de dichos agentes, han surgido diversas arquitecturas como Retrieval Augmented Generation (RAG) y Reasoning and Acting (ReAct). Asimismo, tecnologías como LangChain, LangGraph y LangSmith constituyen marcos de desarrollo para facilitar su aplicación. Entre estas tecnologías, el Model Context Protocol (MCP) emerge con la idea de estandarizar el funcionamiento de los agentes independientemente de su implementación específica.

En este auge tecnológico, LKS Next está aprovechando el potencial de estos sistemas para desarrollar productos que automaticen tareas en producción. No obstante, el carácter innovador requiere un estudio constante de los últimos avances, tarea que el presente Trabajo de Fin de Grado tiene como objetivo facilitar. Mediante la exploración de estas tecnologías en el ámbito del Onboarding, este trabajo tiene como objetivo principal investigar arquitecturas de aplicación útiles y evaluar el uso del protocolo MCP.

El resto del presente documento está estructurado del siguiente modo: el Capítulo \ref{ch:chap2} introduce en detalle tanto el dominio de aplicación como técnico. Los Capítulos \ref{ch:chap3} y \ref{ch:chap4} presentan la planificación del proyecto y los requisitos derivados de este. A continuación, los Capítulos \ref{ch:chap5}, \ref{ch:chap6}, \ref{ch:chap7}, \ref{ch:chap8} y \ref{ch:chap9} detallan el desarrollo del proyecto y el sistema implementado, explicando inicialmente el diseño y arquitectura general, tras lo que se pasa a los agentes concretos y su evaluación. Finalmente, el Capítulo \ref{ch:chap10} presenta el seguimiento y control del proyecto, tras lo que los Capítulos \ref{ch:chap11} y \ref{ch:chap12} exponen los retos y conclusiones finales.
